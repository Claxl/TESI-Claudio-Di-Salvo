\chapter{Conclusioni}
\label{chap:conclusioni}
L'obiettivo di questo elaborato era quello di creare degli strumenti che permettessero l'interfacciamento al \textit{Cloud} e la riprogrammazione completa e parziale dell'\textit{FPGA}.\\
Lungo la trattazione è stato dimostrato come l'integrazione tra il sistema \textit{Cloud} e le \textit{FPGA} sia possibile e come la loro completa riprogrammazione sia effettuabile ed automatizzabile.\\
\\
Purtroppo non è stato possibile raggiungere quest'obiettivo.\\ L'implementazione di un meccanismo per la riconfigurazione parziale non ha riscontrato i risultati attesi, infatti il device scelto per la creazione dell'architettura non supporta la riprogrammazione parziale, In seguito ad un'analisi approfondita del sistema è stato possibile però trovare eventuali soluzioni valide.\\
\\
L'architettura presentata potrebbe espandersi.\\
Per questo motivo questo elaborato fornisce anche le conoscenze base per permettere ad un eventuale sviluppatore di continuare il lavoro svolto.\\
\\
Una possibile soluzione per una prossima implementazione potrebbe essere la creazione di una libreria d'interfacciamento tra \textit{Processing System} e \textit{Programmable Logic}, al fine di permettere la riconfigurazione parziale della scheda.\\
Un altro esempio di  implementazione potrebbe essere la creazione di core proprietari che permettano la definizione delle regioni nella fase di \textit{Floorplanning}.
