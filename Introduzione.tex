\chapter{Introduzione}
\label{chap:intro}
Negli ultimi tempi nel panorama informatico globale è diventato d'uso comune il termine \textit{Cloud} che rappresenta un'infrastruttura di rete remota alla quale, tramite Internet, si è in grado di interfacciarsi. All'interno del \textit{Cloud} è possibile trovare varie risorse come spazi d'archiviazione, indirizzi di rete e macchine virtuali.\\
Con l'avvento dell' \textit{Internet of Things} nei sistemi \textit{Cloud} non è raro trovare anche dispositivi con i quali è possibile interfacciarsi, interconnettersi, acquisire ed elaborare dati. Da questo panorama vengono esclusi dispositivi basati su tecnologia \textit{Field Programmable Gate Array} o \textit{FPGA}, per via della loro integrazione con il sistema \textit{Cloud}.\\
Tale caratteristica è data dalla natura del dispositivo che, a differenza dei sistemi comunemente usati, necessiterà di un percorso diametralmente opposto per la programmazione, composto da un posizionamento dei blocchi logici che formano le \textit{FPGA} ed una traduzione del comportamento logico del sistema.\\
Inoltre, la strumentazione per effettuare la progettazione finale delle \textit{FPGA} è proprietaria ed a sorgente chiuso.\\
L'elaborato si pone come obiettivo quello di creare strumenti, gratuiti ed a codice sorgente aperto, che permettano una più semplice integrazione nei sistemi \textit{Cloud} già esistenti.
In particolare, si è progettato l'interfacciamento del sistema con l'ambiente cloud e la sua completa e parziale riprogrammazione.
L'implementazione di tale sistema è stata testata tramite la \textit{Zedboard}.
La realizzazione dell'ambiente \textit{Cloud} e le politiche di gestione della risorsa rappresentano il lavoro necessario alla finalizzazione del progetto.\\
Nel primo capitolo si discuteranno gli argomenti di base su cui si incentra l'elaborato.\\
Introducendo i dispositivi \textit{FPGA} e le loro funzionalità. In seguito verranno trattati i linguaggi di descrizione dell'hardware, di conseguenza gli strumenti tramite i quali è possibile descrivere il comportamento dell'\textit{FPGA}.\\
Inoltre verrà fornita un'introduzione sulle tecnologie utilizzate nello sviluppo e che saranno necessarie al continuo del lavoro.\\
Nel Secondo capitolo si discuterà la soluzione ideata al fine del raggiungimento degli obiettivi di progetto.\\
Nel Terzo capitolo verrà implementato il sistema operativo per le schede \textit{FPGA}, approfondendo le procedure di programmazione del kernel e descrivendo il workflow necessario al fine di avere un sistema connesso al Cloud.\\
Nel Quarto Capitolo si approfondirà l'interfacciamento interno del \textit{FPGA}, andando a descrivere i vari bus e verrà implementato un driver per l'interfacciamento con il GPIO della scheda tramite il sistema operativo. Questo driver ci permetterà l'acquisizione e l'elaborazione in loco dei dati.\\
Nel Quinto capitolo si discuterà sull'interfaccia necessaria alla riprogrammazione della scheda, sviluppando uno script che permette la \textit{Full Reconfiguration}. Inoltre, verranno discusse le metodologie di riprogrammazione parziale e le sue problematiche.\\
Infine, verrà fornita una guida completa ed integrale alla configurazione e all'uso di tutte le componenti trattate.\\
Per completezza i codici sorgenti inerenti alla trattazione verranno inseriti nella loro totalità nell'appendice, per facilitarne la fruizione.