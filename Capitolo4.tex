\chapter{Gate Array Full Reconfiguration}
\label{chap:Cap3}
Come già anticipato nel capitolo \ref{chap:Cap1}, lo scopo dell'architettura è quello di permettere la personalizzazione completa dell'FPGA al singolo utente, quindi ad esempio l'uso di un nuovo acceleratore, ma senza modificare il kernel linux.\\
Al fine di far ciò è necessario riprogrammare la Programmable Logic tramite la classe, presente nel modulo del kernel, FPGA Manager, cosi da garantire l'uso della risorsa cloud all'utente in base alle sue necessità.
\section{FPGA Manager}
Esso bisogna che sia abilitato nella struttura del kernel, poichè è un insieme di API agnostiche che la loro chiamata ci permette di programmare, o riprogrammare, l'FPGA tramite un file binario.
\subsection{Creazione del file binario}
Per la creazione del file binario, si necessita un bitstream la quale generazione è discussa nell'appendice \ref{ExportVivado}.\\
Questa procedura necessita del tool bootgen, si rimanda a \ref{installazioneBoot} per l'installazione, il quale al fine di generare il file binario necessità uno file intermedio il Boot Image Format (BIF), questo file è cosi strutturato:
\begin{lstlisting}[language=sh, label=lst:C, caption={template file .bif}]
all:{./path/bitstream.bit}
\end{lstlisting}
Il file solitamente contiene tutte le fasi di boot ed eventuali partizioni dell'immagine.\\
Al termine di ciò sarà necessario caricare l'environment di lavoro di bootgen, semplicemente spostandosi nella cartella d'installazione ed eseguendo il seguente:
\begin{lstlisting}[language=sh, label=lst:C, caption={setup environment bootgen}]
source  settings64.sh
\end{lstlisting}
In questo modo tramite il comando 
\begin{lstlisting}[language=sh, label=lst:C, caption={setup environment bootgen}]
bootgen -image [/path/to/bifFile] -arch zynq -process_bitstream bin
\end{lstlisting}
All' appendice \ref{autoBOOT} è disponibile uno script in grado di automatizzare questo processo.
\section{Riprogrammzione Programmable Logic}
Dopo la generazione del file binario, esso andrà copiato nella scheda SD in una qualsiasi partizione.\\
Al fine di abilitare la riprogrammazione completa della scheda sarà sfruttata la classe FPGA Manager, essa per l'abilitazione necessità alcuni comandi da eseguire all'interno della scheda\cite{PL}:
\begin{lstlisting}[language=sh, label=lst:C, caption={Abilitazione FPGA\_manager}]
echo 0 > /sys/class/fpga_manager/fpga0/flags
\end{lstlisting}
Questo comando imposta le Flags dell'fpga manager per accettare un nuovo bistream, quindi di fatto abilitando la riprogrammazione.\\
Di seguito bisognerà caricare il bitstream nella Programmable Logic
\begin{lstlisting}[language=sh, label=lst:C, caption={Caricamento bitstream nella Programmable Logic}]
mkdir -p /lib/firmware
cp [NomeBIN.bin] /lib/firmware/
\end{lstlisting}
Infine per effettuare la completa riprogrammazione è necessario puntare il nuovo file binario caricato nella PL, per far ciò è necessario comunicarlo al FPGA\_Manager
\begin{lstlisting}[language=sh, label=lst:C, caption={Comunicazione nuovo bitstream alla PL}]
echo [NomeBIN.bin] > /sys/class/fpga_manager/fpga0/firmware
\end{lstlisting}
La verifica dell'avvenuta riprogrammazione è visibile tramite un led on-board o tramite l'esecuzione di un sorgente che sfrutta il driver GPIO precedentemente discusso al capitolo \ref{comunicazioneCap}.\\
All'appendice \ref{Riprogrammazione} è possibile trovare uno script automatico per la riprogrammazione.